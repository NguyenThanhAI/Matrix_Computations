\documentclass[14pt, a4paper]{article}
\usepackage{minitoc}
\usepackage[left=3.00cm, right=2.5cm, top=2.00cm, bottom=2.00cm]{geometry}
\usepackage{amsmath}
\usepackage{amssymb}
\usepackage{amsthm}
\usepackage{mathtools}
\usepackage{graphicx}
\usepackage{algpseudocode}
\usepackage{algorithm}
\usepackage{blindtext}
\usepackage{setspace}
\usepackage[utf8]{inputenc}
\usepackage[utf8]{vietnam}
\usepackage[center]{caption}
\usepackage{fancyhdr} % header, footer
\usepackage{hyperref} % loại bỏ border với mục lục và công thức
\pagestyle{fancy}
%\usepackage[style=numeric,sortcites]{biblatex}
%\addbibresource{ref.bib}
%\usepackage[numbers]{natbib}
\usepackage{indentfirst}
\usepackage[natbib,backend=biber,style=ieee, sorting=ynt]{biblatex}
\bibliography{ref.bib}

%\renewbibmacro*{cite}{%
%  \printtext[bibhyperref]{%
%    \printfield{prefixnumber}%
%    \printfield{labelnumber}%
%    \ifbool{bbx:subentry}%
%      {\printfield{entrysetcount}}%
%   \ifnumequal{\value{citecount}}{\value{citetotal}-1}%
%       {\gdef\multicitedelim{\addspace\bibstring{and}\space}}%
%       {\gdef\multicitedelim{\addcomma\space}}%
%    }%
%}

\hypersetup{
    colorlinks=false,
    pdfborder={0 0 0},
}

\title{Tiểu luận phương pháp số cho đại số tuyến tính}

\author{Nguyễn Chí Thanh}

%\date{24-04-2022}
\fancyhf{}
\rhead{\textbf{Học viên thực hiện: Nguyễn Chí Thanh}}
\lhead{\textbf{GVHD: TS. Nguyễn Trung Hiếu}}
\rfoot{\thepage}
\lfoot{\textbf{Phương pháp số cho đại số tuyến tính}}
\renewcommand{\headrulewidth}{0.4pt}
\renewcommand{\footrulewidth}{0.4pt}

\numberwithin{equation}{section}
\numberwithin{algorithm}{section}
\numberwithin{figure}{section}

\setlength{\parindent}{0pt}

\setcounter{secnumdepth}{3} % Cho phép subsubsection trong report
\setcounter{tocdepth}{3} % Chèn subsubsection vào bảng mục lục

\newtheorem{dl}{Định lý}
\newtheorem{md}{Mệnh đề}
\newtheorem{bd}{Bổ đề}

\numberwithin{dl}{section}
\numberwithin{md}{section}
\numberwithin{bd}{section}

\doublespacing
\begin{document}

\begin{titlepage}

    \newcommand{\HRule}{\rule{\linewidth}{0.5mm}} % Defines a new command for the horizontal lines, change thickness here
    
    \center % Center everything on the page
     
    %----------------------------------------------------------------------------------------
    %	HEADING SECTIONS
    %----------------------------------------------------------------------------------------
    \textsc{\LARGE Đại học Quốc Gia Hà Nội}\\[0.5cm]
    \textsc{\LARGE Đại học Khoa học tự nhiên}\\[0.5cm] % Name of your university/college
    \textsc{\LARGE Khoa Toán - Cơ - Tin học}\\[0.5cm]

    \includegraphics[scale=0.2]{figures/HUS-logo.jpg}\\[0.5cm]

    \textsc{\Large Chuyên ngành: Khoa học dữ liệu}\\[0.5cm] % Major heading such as course name

    
    %----------------------------------------------------------------------------------------
    %	TITLE SECTION
    %----------------------------------------------------------------------------------------
    
    \HRule \\[0.4cm]
    { \huge \bfseries Tiểu luận môn học}\\[0.4cm] % Title of your document
    \HRule \\[1.5cm]

    \textsc{\Large Môn học: Phương pháp số cho đại số tuyến tính }\\[1.5cm] % Minor heading such as course title


    \textsc{\Large Đề tài: So sánh thuật toán GMRES và thuật toán MINRES }\\[1.5cm]
     

    %----------------------------------------------------------------------------------------
    %	AUTHOR SECTION
    %----------------------------------------------------------------------------------------
    \begin{minipage}{0.4\textwidth}
        \begin{flushleft} \Large
        \emph{Giảng viên hướng dẫn:} \\
        TS. Nguyễn Trung Hiếu % Supervisor's Name
        \end{flushleft}
    \end{minipage}\\[2cm]

    \begin{minipage}{0.4\textwidth}
    \begin{flushleft} \Large
    \emph{Học viên thực hiện:}\\
    Nguyễn Chí Thanh \\
    MSHV: 21007925 \\ % Your name
    Lớp: Khoa học dữ liệu - K4
    \end{flushleft}
    \end{minipage}
    
    
    % If you don't want a supervisor, uncomment the two lines below and remove the section above
    %\Large \emph{Author:}\\
    %John \textsc{Smith}\\[3cm] % Your name
    
    %----------------------------------------------------------------------------------------
    %	DATE SECTION
    %----------------------------------------------------------------------------------------
    
    % I don't want day because it is English
    % {\large \today}\\[2cm] % Date, change the \today to a set date if you want to be precise
    
    %----------------------------------------------------------------------------------------
    %	LOGO SECTION
    %----------------------------------------------------------------------------------------
    
    %\includegraphics{logo/rsz_3logo-khtn.png}\\[1cm] % Include a department/university logo - this will require the graphicx package
     
    %----------------------------------------------------------------------------------------
    
    \vfill % Fill the rest of the page with whitespace
    
\end{titlepage}

\cleardoublepage
\pagenumbering{gobble}
\tableofcontents
\newpage
\listoffigures
\cleardoublepage
\pagenumbering{arabic}

%\maketitle

\newpage

\nocite{*}

\section{Tổng quan về các phương pháp lặp trên không gian con Krylov}

Các phương pháp trên không gian con Krylov là một nhóm quan trọng trong việc giải các hệ phương trình. Ta xét một phép biến đổi tuyến tính dưới dạng một hộp đen:

\begin{equation}
    x \rightarrow \boxed{\mathrm{black} \thickspace \mathrm{box}} \rightarrow Ax
\end{equation}

Chúng ta muốn xây dựng một phương pháp lặp sao cho $x_n \rightarrow x$ với $Ax=b$ khi $n \rightarrow \infty$. Gọi $x_n \in \mathcal{K}_n$ với $\mathcal{K}_n$ là không gian con Krylov thứ n $\mathcal{K}_n=\mathrm{span} \lbrace b, Ab, A^2b, \dots, A^{n-1}b \rbrace$
Các đặc điểm của không gian con Krylov:
\begin{itemize}
    \item Ta có thể xây dựng $\mathcal{K}_n$ như một hộp đen
    \item $\mathcal{K}_n \subseteq \mathcal{K}_{n+1}$
    \item Giả sử $p^n(A)$ là một đa thức của ma trận $A$. Bất kỳ một tổ hợp tuyến tính của các vector $b, Ab, \dots$ cũng bằng $p^n(A)$ nhân với $b$, $\lVert b - Ax \rVert_2^2 = \lVert p^n(A)b \rVert_2^2$
\end{itemize}

Ta xét ma trận Krylov, $K_n = \begin{bmatrix} b, Ab, A^2b, \dots, A^{n-1}b \end{bmatrix}$. Đây là một ma trận điều kiện tồi. Vì khi $n$ càng lớn, vector $A^nb$ tiến gần đến bội của một vector riêng của ma trận $A$, làm cho ma trận $K_n$ rất gần với một ma trận kỳ dị. Vì vậy ta cần làm việc với một hệ sơ sở trực chuẩn.

Với ma trận $A$ có kích thước $m \times m$, ta có thể tính một ma trận trực giao $Q$ và một ma trận Hessenberg $H$ có dạng một ma trận tam giác trên và một đường chéo phụ:

\begin{equation}
    A = QHQ^T
\end{equation}

$Q_n=\begin{bmatrix} q_1, q_2, \dots, q_n \end{bmatrix}$ là $n$ cột đầu của ma trận $Q$:

\begin{equation} \label{eq:Hessenberg_matrix}
    \widetilde{H}_n = \begin{bmatrix} h_{1,1} & h_{1, 2} & \dots & h_{1, n} \\
    h_{2, 1} & h_{2, 2} & \dots & h_{2, n} \\
    0 & h_{3, 2} & \dots  & h_{3, n} \\
    \vdots & \space & \space & \vdots \\
    0 & \dots & h_{n, n-1} & h_{n, n} \\
    0 & 0 & \dots & h_{n+1, n}\end{bmatrix}
\end{equation}

$\widetilde{H}_n$ là ma trận tam giác trên và một đường chéo phụ kích thước $(n+1)\times n$ với:

\begin{equation} \label{eq:A_projection}
    AQ_n = Q_{n+1}\widetilde{H}_n
\end{equation}

\begin{equation} \label{eq:recurrence_term}
    Aq_n = h_{1, n}q_1 + h_{2, n}q_2 + \dots + h_{n, n}q_n + h_{n+1, n}q_{n+1}
\end{equation}


\subsection{Thuật toán lặp Arnoldi}


\begin{algorithm}
    \caption{Thuật toán Arnoldi}\label{alg:Arnoldi}
    \begin{algorithmic}
        \State {Cho ma trận $A, b$, gán $q_1 \leftarrow b/\lVert b \rVert$}
        \For {$n = 1,2,3,\dots$}
            \State $v \leftarrow Aq_n$
            \For {$j=1$ to $n$}
                \State $h_{j,n} \leftarrow q_j^T v$
                \State $v \leftarrow v - h_{j,n}q_j$
            \EndFor
            \State $h_{n+1,n} \leftarrow \lVert v \rVert$
            \State $q_{n+1} \leftarrow v/h_{n+1,n}$
        \EndFor
    \end{algorithmic}
\end{algorithm}

Công thức \ref{eq:recurrence_term} là công thức truy hồi $n+1$ tham số cho vector $q_{n+1}$. Ma trận \ref{eq:Hessenberg_matrix} thu được từ quá trình trực chuẩn hóa Gram-Schmidt. Quá trình này được gọi là thuật toán Arnoldi được miêu tả ở thuật toán \ref{alg:Arnoldi}. $Q_n=\begin{bmatrix} q_1, q_2, \dots, q_n \end{bmatrix}$ là cơ sở trực chuẩn của không gian con Krylov thứ n $\mathcal{K}_n$.
Nhân cả 2 vế công thức \ref{eq:A_projection} với $Q_n^T$ ta được:

\begin{equation}
    Q_n^TAQ_n=Q_n^TQ_{n+1}\widetilde{H}_n \\
    \Rightarrow Q_n^TQ_{n+1}\widetilde{H}_n = \begin{bmatrix}
        q_1^T \\ q_2^T \\ \vdots \\ q_n^T
    \end{bmatrix} \begin{bmatrix} q_1, q_2, \dots, q_n, q_{n+1} \end{bmatrix}
    \begin{bmatrix} H_n \\ h_{n+1,n}e_n^T \end{bmatrix}=\begin{bmatrix} I & 0 \end{bmatrix} \begin{bmatrix} H_n \\ h_{n+1,n}e_n^T \end{bmatrix}=H_n
\end{equation}

$H_n$ là ma trận thu được bằng cách lấy $n$ hàng đầu của ma trận $\widetilde{H}_n$. $H_n$ có thể được giải thích là biểu diễn trong hệ cơ sở $\lbrace q_1, q_2, \dots, q_n \rbrace$ của phép chiếu trực giao của ma trận $A$ lên $\mathcal{K}_n$, giới hạn ánh xạ $A: \mathbb{C}^m \mapsto \mathbb{C}^m$ sang $H_n: \mathcal{K}_n \mapsto \mathcal{K}_n$. Phép chiếu này được gọi là phép chiếu Rayleigh-Ritz.
Các thành phần nằm trên đường chéo của ma trận $H_n$ chính là các thương số Rayleigh tương ứng với vector $q_j$. Với thương số Rayleigh được tính bằng công thức:

\begin{equation}
    r(x) = \dfrac{x^T A x }{x^T x}
\end{equation}

Các giá trị riêng của ma trận $H_n$ được gọi là ước lượng giá trị riêng Arnoldi (tại bước $n$) hoặc các giá trị Ritz (tương ứng với $\mathcal{K}_n$) của A. Một vài số trong những số này có thể xấp xỉ bằng một vài trong những giá trị riêng của ma trận $A$ ngay cả khi $n$ nhỏ hơn rất nhiều so với $m$

Để sử dụng thuật toán lặp Arnoldi để ước lượng các giá trị riêng, ta tính giá trị riêng của $H_n$ tại bước thứ $n$. Nếu $m=n$, các giá trị Ritz là các giá trị riêng của ma trận $A$. Nói chung, nếu $n \ll m$, chỉ một số ít giá trị riêng được ước lượng. Thường các giá trị riêng có giá trị gần vùng biên của phổ (tập các giá trị riêng) sẽ được hội tụ đầu tiên.

Trong nhiều ứng dụng thực tế, các giá trị tại khu vực gần biên của phổ được quan tâm chủ yếu:

\begin{itemize}
    \item Phân tích tích ổn định thường yêu cầu ước lượng bán kính phổ
    \item Phân tích thành phần chính thường yêu cầu ước lượng giá trị riêng lớn nhất và các vector riêng tương ứng của ma trận $A^T A$
\end{itemize}

Nếu $A$ có $m$ giá trị riêng phân biệt, thuật toán lặp Arnoldi tìm được tất cả các giá trị riêng này sau $m$ bước. Trong một số trường hợp cụ thể, tốc độ hội tụ trong việc ước lượng các giá trị riêng của thuật toán lặp Arnoldi mang tính hình học (tuyến tính, hàm mũ, ...).

Một số những đặc tính bất biến của thuật toán lặp Arnoldi được nêu ở trong định lý \ref{dl:Arnoldi_Properties}

\begin{dl} \label{dl:Arnoldi_Properties}
    Một thuật toán lặp Arnoldi áp dụng cho một ma trận $A \in \mathbb{R}^{m \times m}$ thỏa mãn các tính chất sau:
    \begin{itemize}
        \item \textbf{Bất biến tịnh tiến:} Nếu ma trận $A$ thay đổi thành $A + \sigma I $ với một số $\sigma \in \mathbb{R}$, và $b$ không đổi thì các giá trị Ritz $\lbrace \theta_j \rbrace$ thay đổi thành $\lbrace \theta_j + \sigma \rbrace$
        \item \textbf{Bất biến co dãn:} Nếu ma trận $A$ thay đổi thành $\sigma A$ với một số $\sigma \in \mathbb{R}$ và $b$ không đổi thì $\lbrace \theta_j \rbrace$ thành $\lbrace \sigma \theta_j \rbrace$
        \item \textbf{Bất biến dưới phép biến đổi tương đương trực giao:} Nếu ma trận $A$ thay đổi thành $UAU^T$ với một ma trận $U$ trực giao bất kỳ, $b$ thay đổi thành $Ub$ thì $\lbrace \theta_j \rbrace$ không đổi
    \end{itemize}
    Trong cả ba trường hợp trên, các vector Ritz $Q_n y_n$ tương ứng với vector riêng $y_j$ của ma trận $H_n$ không đổi dưới bất kỳ phép biến đổi nào đã được nêu trên.
\end{dl}

Một trường hợp đặc biệt, trong quá trình thực hiện thuật toán Arnoldi, tại một bước $n$ nào đó khi tính $v \leftarrow v - h_{j,n}q_j$, nếu thu được $v=0$ ta có thể dừng luôn thuật toán Arnoldi vì không gian con Krylov đã mở rộng đến cực đại, các điều sau đồng thời xảy ra:


\begin{itemize}
    \item $A\mathcal{K}_n \subseteq  \mathcal{K}_n$
    \item $\mathcal{K}_n=\mathcal{K}_{n+1}=\mathcal{K}_{n+2}=\dots$
    \item Từng giá trị riêng của$ H_n$ là giá trị riêng của $A$
    \item Nếu ma trận $A$ là ma trận không kỳ dị, nghiệm chính xác của phương trình $Ax=b$ nằm trong $\mathcal{K}_n$
\end{itemize}


Thuật toán lặp Arnoldi thuận tiện cho việc xây dựng cơ sở cho không gian con Krylov $\mathcal{K}_{n+1}$ từ cơ sở $\mathcal{K}_n$, mỗi bước này tạo thêm một vector mới $q_{n+1}$ và tránh việc sử dụng vector $A^n b$

Thuật toán Arnoldi được sử dụng trong các trường hợp:
\begin{itemize}
    \item Là cơ sở cho các thuật toán lặp giải hệ phương trình (GMRES, ...).
    \item Kỹ thuật tính giá trị riêng của ma trận không đối xứng.
\end{itemize}

\subsection{Thuật toán lặp Lanczos}


Nếu $A$ là ma trận đối xứng thì ma trận $\widetilde{H}_n$ trở thành ma trận ba đường chéo $\widetilde{T}_n$ có dạng:

\begin{equation} \label{eq:Trigonal_Matrix}
    \widetilde{T}_n = \begin{bmatrix}
        \alpha_1 & \beta_1 & \space & \space & \space \\
        \beta_1 & \alpha_2 & \beta_1 & \space & \space \\
        \space & \beta_2 & \alpha_3 & \ddots & \space \\
        \space & \space & \ddots & \ddots & \beta_{k-1} \\
        \space & \space & \space & \beta_{k-1} & \alpha_k \\
        \space & \space & \space & \space & \beta_{n}
    \end{bmatrix}
\end{equation}

Ma trận $\widetilde{T}_n$ thu được từ thuật toán Lanczos được miêu tả ở thuật toán \ref{alg:Lanczos}
Với:

\begin{equation}
    AQ_n = Q_{n+1}\widetilde{T}_n \Rightarrow Q_n^TAQ_n = Q_n^TQ_{n+1}\widetilde{T}_n=\begin{bmatrix}
        q_1^T \\ q_2^T \\ \vdots \\ q_n^T
    \end{bmatrix} \begin{bmatrix} q_1, q_2, \dots, q_n, q_{n+1} \end{bmatrix}\begin{bmatrix}
        T_n \\ \beta_n e_n^T
    \end{bmatrix}=\begin{bmatrix}
        I & 0
    \end{bmatrix} \begin{bmatrix}
        T_n \\ \beta_n e_n^T
    \end{bmatrix}=T_n
\end{equation}

$T_n$ là ma trận thu được bằng cách lấy $n$ hàng đầu của ma trận $\widetilde{T}_n$. 
Ma trận $T_n$ về mặt hình thức còn được gọi là ma trận truy hồi ba tham số.

\begin{algorithm}
    \caption{Thuật toán Lanczos}\label{alg:Lanczos}
    \begin{algorithmic}
        \State {Cho ma trận $A$ đối xứng, $b$, gán $\beta_0 \leftarrow 0, q_0 \leftarrow 0, q_1 \leftarrow b/\lVert b \rVert$}
        \For {$n=1,2,3,\dots$}
            \State $v \leftarrow Aq_n$
            \State $\alpha_n=q_n^Tv$
            \State $v \leftarrow v - \beta_{n-1}q_{n-1} - \alpha_n q_n$
            \State $\beta_n \leftarrow \lVert v \rVert$
            \State $q_{n+1}=v/\beta_n$
        \EndFor
    \end{algorithmic}
\end{algorithm}

Từ thuật toán \ref{alg:Lanczos}, ta nhận thấy mỗi bước của thuật toán Lanczos, bao gồm một phép nhân ma trận, một phép tích vô hướng và hai phép trừ vector, ít hơn so với thuật toán Arnoldi (1 phép nhân ma trận, $n$ phép tích vô hướng và $n$ phép trừ vector)
Thuật toán đặc biệt hiệu quả cho những ma trận thưa. Trong thực tế, thuật toán lặp Lanczos hay được sử dụng để tính các giá trị riêng của các ma trận đối xứng có kích thước lớn.

Cũng giống như thuật toán lặp Arnoldi, thuật toán lặp Lanczos cũng rất hữu ích và hay được sử dụng cho các bài toán:

\begin{itemize}
    \item Cơ sở cho các thuật toán lặp (ví dụ MINRES, Gradient liên hợp)
    \item Một kỹ thuật dùng để ước lượng giá trị riêng của các ma trận đối xứng
\end{itemize}

Khi ước lượng giá trị riêng của một ma trận đối xứng sử dụng thuật toán lặp Lanczos, các giá trị Ritz thường hội tụ về các giá trị riêng nằm ở gần vùng biên của phổ đầu tiên. 

Một vấn đề khá lớn của thuật toán lặp Lanczos là ảnh hưởng của sai số làm tròn.
Sai số làm tròn có một tác động phức tạp lên quá trình thực hiện thuật toán lặp Lanczos. Thực tế trong đại số tuyến tính số học, tất cả các phép lặp dựa trên truy hồi ba tham số đều chịu ảnh hưởng lớn từ sai số làm tròn.
Nếu trong các phương pháp truy hồi $n$ tham số (ví dụ thuật toán lặp Arnoldi), các vector $q_1, q_2, q_3, \dots$ bị bắt buộc trở thành trực giao bởi thủ tục trực chuẩn hóa Gram-Schmidt. Các thuật toán lặp truy hồi ba tham số (ví dụ thuật toán lặp Lanczos), phụ thuộc vào tính trực giao của các vector $\lbrace q_j \rbrace$, các vector này được sinh ra dựa vào bản chất toán học nhưng bản chất này không được bảo toàn một cách nguyên vẹn trong quá trình tính toán do sai số làm tròn. Vì vậy, sau một số bước lặp tính trực giao của các vector bị biến mất.

Khi tính trực giao của các vector $\lbrace q_j \rbrace$ mất đi, vẫn đề này sẽ làm ảnh hưởng đến quá trình hội tụ của các giá trị Ritz đến các giá trị riêng của ma trận $A$. Nhiều giá trị riêng trong số những giá trị riêng của ma trận $A$ mà mỗi giá trị riêng này có nhiều giá trị Ritz cùng hội tụ đến. Những giá trị Ritz bổ sung thêm một giá trị Ritz ban đầu cùng hội tụ đến một giá trị riêng của ma trận $A$ được gọi là các giá trị riêng "ma".
Một phân tích chặt chẽ về hiện tượng giá trị riêng "ma" rất phức tạp. Tuy nhiên một giải thích trực quan cho hiện tượng này là:

\begin{itemize}
    \item Sự hội tụ của các giá trị Ritz triệt tiêu đi các thành phần của vector riêng tương ứng trong vector đang được vận hành.
    \item Với sai số làm tròn, các nhiễu ngẫu nhiên tái kích hoạt lại các thành phần bị triệt tiêu ở trên, khiến cho một giá trị riêng nào đó của ma trận $A$ được xuất hiện nhiều lần.
\end{itemize}

Hệ quả của hiện tượng này là chúng ta không thể tin tưởng vào số bội của các giá trị riêng của ma trận $A$ trong quá trình ước lượng sử dụng thuật toán lặp Lanczos.
Tuy nhiên thuật toán lặp Lanczos vẫn có thể rất hữu hiệu trong các bài toán thực tế:

\begin{itemize}
    \item PCA để giảm chiều trong phân tích dữ liệu, chúng ta chỉ cần tìm giá trị kỳ dị lớn nhất và vector kỳ dị tương ứng của ma trận $A$.
    \item Một cách tiếp cận chuẩn tắc là áp dụng thuật toán lặp Lanczos vào ma trận $A^T A$ và $A A^T$ mà không tính hai tích trên một cách rõ ràng, sử dụng các vector Ritz để xác định các vector kỳ dị.
\end{itemize}


\section{Thuật toán GMRES}

\subsection{Giới thiệu thuật toán GMRES}

Phương pháp GMRES áp dụng cho hệ phương trình $Ax=b$ có ma trận $A$ là ma trận không kỳ dị. Ta gọi nghiệm chính xác của bài toán là $x^* = A^{-1}b$. Ý tưởng của thuật toán GMRES khá đơn giản, tại bước thứ $n$, ta xác định một vector $x_n \in \mathcal{K}_n$ làm cực tiểu hóa phần dư $r_n = b - A x_n$. Hay nói theo cách khác chúng ta đang tìm $x_n$ giải một bài toán cực tiểu hóa như ở hình \ref{fig:GMRES-LS}



Ta áp dụng công thức
\begin{algorithm}
    \caption{Thuật toán GMRES}\label{alg:GMRES}
    \begin{algorithmic}
        \State{Gán $q_1 \leftarrow b/\lVert b \rVert$}
        \For {$n=1,2,3,\dots$}
            \State {Thực hiện bước thứ n của thuật toán \ref{alg:Arnoldi}}
            \State {Tìm $y$ để cực tiểu hóa $\lVert \beta e_1 - \widetilde{H}_n y \rVert_2=\lVert r_n \rVert_2$}
            \State {$x_n \leftarrow Q_n y$}
        \EndFor
    \end{algorithmic}
\end{algorithm}

\begin{figure}[h!] \centering

    \includegraphics[scale=0.8]{figures/GMRES-LS.jpg}
    \caption{Tìm một vector trong không gian $A \mathcal{K}_n$ \\ sao cho khoảng cách từ điểm này đến vector b nhỏ nhất}

    \label{fig:GMRES-LS}
\end{figure}

Ta xét ma trận Krylov $K_n = \begin{bmatrix} b, Ab, A^2b, \dots, A^{n-1}b \end{bmatrix}$, vì vậy:

\begin{equation}
    A K_n = \begin{bmatrix} Ab, A^2b, A^3b, \dots, A^nb \end{bmatrix}
\end{equation}

Bài toán của chúng ta là tìm một vector $c \in \mathbb{R}^n$ sao cho:

\begin{equation} \label{eq:GMRES-LS-c}
    c = \underset{c \in \mathbb{R}^{n}}{\mathrm{argmin}} \lVert b - AK_n c \rVert
\end{equation}

Với $\lVert \thickspace \rVert$ mặc định là $\lVert \thickspace \rVert_2$. Ta có thể sử dụng phân tích QR với ma trận $AK_n$. Khi đã tìm được $c$ là nghiệm của bài toán cực tiểu hóa \ref{eq:GMRES-LS-c}, $x_n=K_n c$. Nhưng bài toán trên là một bài toán không ổn định (do khi $n$ lớn ma trận $K_n$ rất gần với một ma trận kỳ dị), nên chúng ta sẽ sử dụng thuật toán lặp Arnoldi được miêu tả ở thuật toán \ref{alg:Arnoldi}
để xây dựng một dãy ma trận $Q_n$ mà các vector cột $q_1, q_2, q_3, \dots$ tạo nên cơ sở trực chuản của không gian con Krylov $\mathcal{K}_n$. Vì $x_n \in \mathcal{K}_n$, ta có thể viết $x_n = Q_n y$, bài toán cực tiểu hóa của chúng ta là tìm vector $y \in \mathbb{R}^n$, ta viết lại bài toán \ref{eq:GMRES-LS-c} dưới dạng:

\begin{equation} \label{eq:GMRES-LS-y}
    y = \underset{y \in \mathbb{R}^{n}}{\mathrm{argmin}} \lVert b - A Q_n y \rVert
\end{equation}

Ta sử dụng công thức \ref{eq:A_projection}, bài toán \ref{eq:GMRES-LS-y} có dạng:

\begin{equation}
    y = \underset{y \in \mathbb{R}^{n}}{\mathrm{argmin}} \lVert b - Q_{n+1} \widetilde{H}_n y \rVert
\end{equation}

Vì nhân một vector với một ma trận trực giao không làm thay đổi chuẩn của vector. Ta nhân các vector trong dấu $\lVert \thickspace \rVert$ với ma trận $Q_{n+1}^T$ ta được:

\begin{equation}
    y = \underset{y \in \mathbb{R}^{n}}{\mathrm{argmin}} \lVert Q_{n+1}^T b - Q_{n+1}^T Q_{n+1} \widetilde{H}_n y \rVert
\end{equation}

Trong quá trình thực hiện thuật toán lặp Arnoldi được miêu tả ở thuật toán \ref{alg:Arnoldi} để xây dựng cơ sở trực chuẩn $Q_n$ của không gian con Krylov $\mathcal{K}_n$, ta đặt $\lVert b \rVert = \beta, b=\lVert b \rVert e_1=\beta e_1$ với $e_1 = \begin{bmatrix}
    1 & 0 & 0 & \dots
\end{bmatrix}^T$, ta thu được dạng cuối cùng của bài toán cực tiểu hóa GMRES:

\begin{equation}
    y = \underset{y \in \mathbb{R}^{n}}{\mathrm{argmin}} \lVert \beta e_1 - \widetilde{H}_n y \rVert
\end{equation}

Tại mỗi bước $n$, ta giải bài toán cho $y$ và thu được $x_n = Q_n y$. Nhưng trong thực tế, tại mỗi bước $n$ ta không cần giải tường minh $y$ mà chỉ cần tính phần dư $\lVert \beta e_1 - \widetilde{H}_n y \rVert$, nếu phần dư này đã đủ nhỏ hơn một ngưỡng cho trước, ta sẽ dừng tại bước $n$ hiện tại và tính $x_n$. Nếu phần dư đủ nhỏ ta thực hiện bước $n+1$. Chi tiết cách tính phần dư tại mỗi bước sẽ được trình bày chi tiết ở mục sau.

Các bước cơ bản của thuật toán GMRES được trình bày ở thuật toán \ref{alg:GMRES}

Thuật toán GMRES cũng giải một bài toán xấp xỉ đa thức. Ta xét nghiệm $x_n$, tại bước $n$, vì $x_n \in \mathcal{K}_n$ nên ta có thể viết:

\begin{equation}
    x_n = c_0 b + c_1 A b + c_2 A^b + \dots + c_{n-1}A^{n-1}b=(c_0 + c_1 A + c_2 A^2 + \dots + c_{n-1} A^{n-1})b
\end{equation}

Ta đặt:

\begin{equation}
    q(z) = c_0 + c_1 z + c_2 z^2 + \dots + c_{n-1}z^{n-1}
\end{equation}

Như vậy, nghiệm tại bước thứ $n$ có thể viết dưới dạng đa thức:

\begin{equation}
    x_n = q(A)b
\end{equation}

$z(z)$ là một đa thức có bậc tối đa bằng $n-1$ với các vector hệ số $c$. Phần dư tương ứng $r_n=b - Ax_n=(I - Aq(A))b$, ta đặt đa thức $p_n(z) = 1 - z q(z)$. Như vậy ta thu được:

\begin{equation}
    r_n = p_n(A)b
\end{equation}

Thuật toán GMRES tại mỗi bước tìm một đa thức $p_n \in P_n$ với $P_n$ là:

\begin{equation}
    P_n = \lbrace \text{là tập các đa thức có bậc } \leq n \text{ với } p(0)=1 \rbrace
\end{equation}

sao cho $p_n$ làm cực tiểu hóa phần dư (tìm vector hệ số $c$ của $p_n$):

\begin{equation}
    p_n = \underset{p_n \in P_n}{\mathrm{argmin}} \lVert p_n(A)b \rVert
\end{equation}

Thuật toán GMRES cũng thỏa mãn một số tính chất bất biến như thuật toán lặp Arnoldi

\begin{dl}
    Thuật toán GMRES áp dụng cho một ma trận $A \in \mathbb{R}^{m \times m}$ thỏa mãn các tính chất sau:

    \begin{itemize}
        \item \textbf{Bất biến co dãn:} Nếu ma trận $A$ thay đổi thành $\sigma A$ với một số $\sigma \in \mathbb{R}$, và $b$ thay đổi thành $\sigma b$ thì phần dư $\lbrace r_n \rbrace$ thay đổi thành $\lbrace \sigma r_n \rbrace$.
        \item \textbf{Bất biến dưới phép biến đổi tương đương trực giao:} Nếu ma trận $A$ thay đổi thành $A A U^T$ với một ma trận $U$ trực giao bất kỳ, và $b$ thay đổi thành $Ub$, thì phần dư $\lbrace r_n \rbrace$ thay đổi thành $\lbrace U r_n \rbrace$.
    \end{itemize}
\end{dl}

Thuật toán GMRES không có tính chất bất biến tịnh tiến như thuật toán lặp Arnoldi, vì điều kiện chuẩn hóa $p(0)=1$ liên quan đến sự phụ thuộc quá trình tịnh tiến. Sự thay đổi của $\lbrace r_n \rbrace$ theo phép tịnh tiến phụ thuộc nhiều vào lựa chọn điểm gốc

\subsection{Chi tiết thực hiện thuật toán GMRES}



\section{Thuật toán MINRES}

Phương pháp MINRES áp dụng cho hệ phương trình $Ax=b$ có ma trận $A$ là ma trận đối xứng.

\begin{algorithm}
    \caption{Thuật toán MINRES}\label{alg:MINRES}
    \begin{algorithmic}
        \State{Gán $q_1 \leftarrow b/\lVert b \rVert$}
        \For {$n=1,2,3,\dots$}
            \State {Thực hiện bước thứ n của thuật toán \ref{alg:Lanczos}}
            \State {Tìm $y$ để cực tiểu hóa $\lVert \beta e_1 - \widetilde{T}_n y \rVert_2=\lVert r_n \rVert_2$}
            \State {$x_n \leftarrow Q_n y$}
        \EndFor
    \end{algorithmic}
\end{algorithm}


\section{So sánh phương pháp GMRES và phương pháp MINRES}

Phương pháp GMRES và MINRES là các phương pháp lặp trên không gian con Krylov để giải hệ phương trình $Ax=b$ trong các trường hợp tổng quan hơn phương pháp Gradient liên hợp.
MINRES được sử dụng để giải các hệ phương trình mà ma trận $A$ là ma trận đối xứng, còn phương pháp GMRES để giải các hệ phương trình mà ma trận $A$ là ma trận không kỳ dị.
Hai phương pháp này gần giống với phương pháp Gradient liên hợp nhưng phương pháp Gradient liên hợp cực tiểu hóa hàm mục tiêu $\displaystyle \min_{x \in \mathcal{K}_n}\dfrac{1}{2}x^TAx - x^Tb$,
trong khi hai phương pháp GMRES và MINRES tìm vector trong không gian con Krylov thứ n $\displaystyle \mathcal{K}_n$ mà làm cực tiểu hóa phần dư $\displaystyle \min_{x \in \mathcal{K}_n}\dfrac{1}{2}\lVert b - Ax \rVert_2^2$

Hai phương pháp GMRES và MINRES giải các bài toán tối ưu hóa có ràng buộc, nói chung là bài toán khá khó giải. Xét $x_n \in \mathcal{K}_n$, ta có thể biểu diễn $x_n$ dạng:
$x_n = Q_n y$ với $y$ là tọa độ của $x_n$ trong hệ cơ sở trực chuẩn của không gian Krylov thứ n $\mathcal{K}_n$. Hàm mục tiêu trở thành:
\begin{equation} \min_{y \in \mathbb{R}^{n}}  \lVert b - AQ_n y \rVert_2^2\end{equation}

Như vậy bài toán đã được đưa về bài toán tối ưu có ràng buộc với số chiều của $y$ nhỏ hơn so với $x$. Tuy nhiên với ma trận $A$ có kích thước lớn thì bài toán trên vẫn khá đắt.

Nhưng chúng ta có thể giảm số chiều của bài toán đi nữa bằng cách nhân hàm mục tiêu với ma trận trực giao $Q_{n+1}$ là ma trận cơ sở trực chuẩn của không gian con Krylov thứ n+1:
\begin{equation} \min_{y \in \mathbb{R}^{n}}  \lVert b - AQ_n y \rVert_2^2 = \min_{y \in \mathbb{R}^{n}}  \lVert Q_{n+1}^T(b - AQ_n y) \rVert_2^2 \label{eq:uncons_obj}\end{equation}

Ta sử dụng công thức:

\begin{equation}
    AQ_n = Q_{n+1}\widetilde{H}_n
\end{equation}

đối với phương pháp GMRES, đối với phương pháp MINRES:
\begin{equation}
    AQ_n = Q_{n+1} \widetilde{T}_n
\end{equation}
trong đó, $\widetilde{H}_n$ là ma trận Hessenberg có dạng ở công thức \ref{eq:Hessenberg_matrix}:
và ma trận $\widetilde{T}_n$ là ma trận 3 đường chéo được đề cập ở công thức \ref{eq:Trigonal_Matrix}:

Với $\lVert b \rVert = \beta, q_1 = \dfrac{b}{\beta}$, hàm mục tiêu \eqref{eq:uncons_obj} trở thành:

\begin{equation}
    \min_{y \in \mathbb{R}^{n}}  \lVert Q_{n+1}^T(b - AQ_n y) \rVert_2^2=\min_{y \in \mathbb{R}^{n}} \lVert \beta e_1 - \widetilde{H}_n y\rVert_2^2
\end{equation}

đối với phương pháp GMRES (ma trận A là ma trận không suy biến).
Đối với phương pháp MINRES (ma trận A là ma trận đối xứng):
\begin{equation}
    \min_{y \in \mathbb{R}^{n}}  \lVert Q_{n+1}^T(b - AQ_n y) \rVert_2^2=\min_{y \in \mathbb{R}^{n}} \lVert \beta e_1 - \widetilde{T}_n y\rVert_2^2
\end{equation}

Như vậy, bài toán được đưa về một bài toán bình phương tối thiểu với kích thước $(n+1)\times n$

\begin{equation}
    y_n = \operatorname*{argmin}_{y_n \in \mathbb{R}^{n}} \lVert \beta e_1 - \widetilde{H}_n y\rVert_2^2
\end{equation}

đối với phương pháp GMRES, và:
\begin{equation}
    y_n = \underset{y_n \in \mathbb{R}^{n}}{\mathrm{argmin}} \lVert \beta e_1 - \widetilde{T}_n y\rVert_2^2
\end{equation}

đối với phương pháp MINRES


\newpage
\addcontentsline{toc}{section}{Tài liệu tham khảo}
%\bibliographystyle{IEEEtraN}
%\bibliography{ref}
%\pagestyle{plain}
\printbibliography[title={Tài liệu tham khảo}]

%\newpage
%\printbibliography

\end{document}
